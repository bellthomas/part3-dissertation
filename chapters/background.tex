
\paragraph{} This chapter will cover a number of topics essential to understanding the rationale and implementation of the design as discussed in §~\ref{sec:design}. These include; Intel SGX, a brief overview of modern \textit{libOSes}, an introduction to \textit{Information Flow Control} (\textit{IFC}), and a overview of key aspects of the Linux kernel relevant to the architecture of the prototype.

\section{Intel SGX}

\paragraph{} Intel's Software Guard Extensions, SGX, was first announced and detailed in a handful of whitepaper documents published in 2013. [X,Y,Z] It described a novel approach, creating in-CPU containers with their own protected memory pools. These regions, called \textit{enclaves}, cannot be read from or written due to fundamental protection mechanisms provided by the x86 architecture. \textit{Enclaves} provide both integrity and secrecy to the operation running inside of it, even in the prescence of a malicious host.

\paragraph{Motivation} At a high level SGX aims to achieve security for sensitive application by shielding them, and the resources it uses, against tampering and to provide a guarantee to end users about an enclave's integrity; this is achieved using measurement and attestation. A driving use case is in a cloud computing context, where users are forced to trust a foreign party with both their data and business logic. By distributing encrypted, yet executable, containers targetting a single, unique SGX CPU, users can be assured that their information is safe, regardless of any virutalisation that may be taking place. Only the provisioned CPU is able to decrypt and execute the enclave, strictly in accordance with the restrictions of the SGX platform.

\subsection{Security Characteristics}

\begin{figure}[]
    \centering
    \includegraphics[width=0.9\linewidth]{figures/SGX-architecture.pdf}
    \caption{Abstract overview of SGX's protection in an adversarial environment.}
    \label{fig:sgx-basic}
\end{figure}

\paragraph{} At its heart SGX is designed to be \textit{trustworthy}; this is achieved in a number of ways, including robust enclaving provisioning, sealing and attestation. Intel enumerates SGX's protections as follows;

\begin{itemize}
    \item Memory security against observation and modification from outside the enclave; this is achieved using an in-die \textit{Memory Encryption Engine} (\textit{MEE}), with a secret that rotates on every boot. This protection notably works against a host hypervisor, other enclaves, and anything running in supervisor mode.
    \item Attestation of an enclave to a challenger through the use of a permanent hardware security key for asymmetric encryption.
    \item Proxied software calls to prepare and transfer control in and out of an enclave. Arguments are securely marshalled according to a static enclave definition.
    \item SGX does not defend against reverse engineering or sidechannel attacks: [X,Y] this is the responsibility of the developer to mitigate.
    \item Debugging support is only provided via a specialised tool and only when an enclave is compiled with debugging enabled.
\end{itemize}

\subsection{Architecture and Implementation}

\begin{figure}[]
    \centering
    \includegraphics[width=0.9\linewidth]{figures/SGX-AdvArchitecture.pdf}
    \caption{A high-level overview of the SGX hardware and software architecture.}
    \label{fig:sgx-advarch}
\end{figure}

\paragraph{} The SGX platform comprises a number of interlocking parts, as shown in Figure~\ref{fig:sgx-advarch}. Working from the hardware up, at the heart of the platform is the extended x86 instruction set and memory protection provided by an SGX-capable CPU. 

\paragraph{Hardware} Enclaves' data and code is stored securely in \textit{Processor Reserved Memory}, \textit{PRM}; this is a set of pages in system memory that are presided over by the \textit{MEE}. DMA\footnote{Direct Memory Access} to \textit{PRM} is always rejected. \textit{PRM} consists of two datastructures; the \textit{Enclave Page Cache Map} (\textit{EPCM}) and the \textit{Enclave Page Cache} (\textit{EPC}). An individual enclave is defined by an \textit{SGX Enclave Control Structure}, \textit{SECS}; this is generated when an enclave is created and stored in a dedicated entry in the \textit{EPC}. An enclave's \textit{SECS} contains important information such as its (system) global identifier, its measurement hash and the amount of memory it is using. Access control information is stored in the \textit{EPCM} alongside page validity flags, the owning enclave identifier and the page's type; this is not accessible from software. An attempt to resolve a page in \textit{PRM} is successful only if the CPU is executing in enclave mode and its \textit{EPCM} entry states it belongs the currently executing enclave --- if this is not the case the lookup returns an unused page from generic system memory.

\paragraph{} The host OS or hypervisor manages the \textit{EPC} just as it does with normal system memory, swapping pages in and out according to its own policy, but must do so using SGX specific instructions. The \textit{MEE} is responsible for ensuring the integirty and confidentiality of this process, encrypting and decrypting pages as they cross the \textit{PRM} boundary. Data is verified with the use of an integrity tree, and encryption keys are generated at boot-time. Importantly the SGX archicture relies on the host OS being SGX-aware, empowering userspace applications to function without privilege; this is provided by the SGX driver, \textit{isgx}.

\paragraph{Userspace services} Starting an enclave requires retrieving a \textit{launch token} from Intel's \textit{Launch Enclave}; this checks the signature and identity of the enclave to ensure it is valid. Access to the \textit{Launch Enclave} and other architectural enclaves is provided by the AESM service; the userspace SGX libraries facilitate the communication mechanism. Other architectural enclaves include;

\begin{itemize}
    \item The \textit{Provisioning Enclave} --- this verifies the authenticity of the platform and retrieves an enclave's \textit{attestation key} from the \textit{Intel Provisioning Service's} servers.
    \item The \textit{Quoting Enclave} --- this provides trust in the identity of the SGX environment and enclave being attested, by converting the locally generated \textit{attestation key} to a remotely-verifiable \textit{quote}.
\end{itemize}

\paragraph{Third-party enclaves} Enclaves can only be entered via userspace, as detailed in §~\ref{sec:sgx-lifecycle}, but are always accompanied by a host application which acts as its untrusted counterpart. The host application calls the SGX SDK to build an enclave on its behalf using an enclave image, packaged as a standard shared library (\texttt{enclave.so}) and returns its \textit{global identifier}. Control is passed from the host application to the enclave by invoking an enclave function via an \textit{ECALL}. Execution flow can temporarily leave the enclave if it call one of the host application's function via an \textit{OCALL}. Execution naturally leaves enclave-mode when the \textit{ECALL} terminates. Both \textit{ECALLs} and \textit{OCALLs} are defined statically in the enclave's interface definition (\texttt{enclave.edl}). The necessary glue code is generated by the SGX SDK's build toolchain at compile time; this ensures calls crossing the enclave boundary are marshalled safely and correctly.


\subsection{Enclave Lifecycle}
\label{sec:sgx-lifecycle}

\paragraph{} SGX instructions can be separated into two distinct groups; privileged and unprivileged. These, alongside a description of the function they perform, are enumerated in Table~\ref{table:sgx-instructions}.\footnote{A handful of instructions not relevant to the explanation given here are omitted.} The following description of the process of creating an enclave is illustrated in Figure~\ref{fig:sgx-enclavecreate}.

\paragraph{Preparing an enclave} Execution begins with the host application; this needs to initiate the creation process, but must do so via a component with \textit{Ring 0} privileges. This facilities is provided by \textit{isgx}, the SGX driver. The application first requests \textit{isgx} allocates the necessary number of pages to run the enclave $\langle 1 \rangle$;\footnote{These numbers correspond to events in Figure~\ref{fig:sgx-enclavecreate}.} this is tracked and served from the drivers internal state $\langle 2 \rangle$, as the size of the \textit{EPC} is fixed on boot.

\paragraph{} The application continues to execute \texttt{ECREATE} with the metadata of the enclave to be loaded $\langle 3 \rangle$; the \textit{MEE} checks that the pages being claimed are in fact vacant and populates the \textit{SECS} page with the necessary information $\langle 4 \rangle$. Once this is complete the application preapre the remaining \textit{EPC} pages using \texttt{EADD} $\langle 5 \rangle$a and loads the enclave's code and data $\langle 6 \rangle$.

\paragraph{} At this point the enclave needs to be measured --- the application calls \texttt{EEXTEND} $\langle 7 \rangle$, triggering the \textit{MEE} to update the measurement hash in the \textit{SECS} to aligns with the current state of the enclave's memory $\langle 8 \rangle$. Once the \textit{EPC} memory is prepared the applications requests for it to be finalised using \texttt{EINIT} $\langle 9 \rangle$: this operation requires the application to retrieve the \textit{EINITTOKEN} from the \textit{Launch Enclave}, locking the execution of the measured enclave to the CPU the token is generated on. Notably, pages cannot be added after \texttt{EINIT},\footnote{This is only strictly true in SGXv1, as explained in §~\ref{sec:sgx-versions}.} and an enclave cannot be attested to or entered before it. Finally, the initialised flags is set in the \textit{SECS} and the enclave's hash updated for the final time $\langle 10 \rangle$.



\begin{table}
    \centering
    \newcommand\tableTop{\rule{0pt}{3ex}}
    \newcommand\tableMid{\rule{0pt}{3ex}}
    \newcommand\tableBottom{\rule[-2ex]{0pt}{0pt}}
    \begin{tabular}{|c|r|p{8.5cm}|} 
        \hline
        Execution Mode & Instruction & Function \\ [0.1ex] 
        \hline\hline
        \multirow{11}{*}{Ring 0} 
            & \tableTop{\texttt{ECREATE}} & \tableTop{Generate and copy the \textit{SECS} structure to a new page in the \textit{EPC}, initialising a new enclave.} \\ 
            & \texttt{EADD} & \tableMid{Add a new \textit{EPC} page for the current enclave; this is used to load initial code and data.} \\ 
            & \texttt{EEXTEND} & \tableMid{Updates the enclave's measurement during attestation; modifies the \textit{SECS}.} \\ 
            & \texttt{EINIT} & \tableMid{The terminal instruction in an enclave's initialisation, finalising its attributes and measurement.} \\ 
            & \texttt{EREMOVE} & \tableMid{Permanently remove a page from the \textit{EPC}; usually invoked during enclave destruction.} \tableBottom \\ 
        \hline\hline
        \multirow{11}{*}{Ring 3} 
        & \tableTop{\texttt{EENTER}} & \tableTop{Transfer control from the host application to a pre-determined location in an enclave.} \\ 
        & \texttt{ERESUME} & \tableMid{Re-enter the enclave after an \textit{OCALL} and resume execution.} \\ 
        & \texttt{EEXIT} & \tableMid{Restore the original operating mode at the location \texttt{EENTER} was triggered and flush the TLB.} \\ 
        & \texttt{EGETKEY} & \tableMid{Access platform cryptography keys required for attestation and sealing.} \\ 
        & \texttt{EREPORT} & \tableMid{Generate a \textit{report} for an enclave's \textit{attestation key} for an attestation process.} \tableBottom \\ 
        \hline
    \end{tabular}
    \vspace{5mm}
    \caption{Overview of notable SGX x86 instructions in an enclave's lifecycle.}
    \label{table:sgx-instructions}
\end{table}

\begin{figure}[]
    \centering
    \includegraphics[width=\linewidth]{figures/SGX-EnclaveCreate.pdf}
    \caption{The process of creating and initialising an enclave; details given in §~\ref{sec:sgx-lifecycle}. Purple components are a part of the SGX platform. Diagram inspired by Adamski~[X]}
    \label{fig:sgx-enclavecreate}
\end{figure}


\subsection{Attestation}

\subsection{Provisioning and Sealing}

\subsection{SGX Versions}
\label{sec:sgx-versions}

\section{Modern \textit{libOSes}}

\section{Information Flow Control}

\section{Aspects of the \textit{Linux} Kernel}