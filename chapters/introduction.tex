


\paragraph{} Defending computer systems against malicious programs and enforcing the isolation of protected components has always been exceedingly challenging. A system's \textit{\acrfull{tcb}}, defines the minimal set of components critical to establish and maintain system security and integrity. This traditionally includes, amongst others; the \acrshort{os} kernel; device drivers; device firmware; and hardware. Compromise of a trusted component inside a system's \textit{\acrshort{tcb}} directly threatens any secure application. One approach to hardening a system's security is to minimise its \textit{\acrshort{tcb}}, diminishing its potential \textit{attack surface}. 

\paragraph{} An increasingly common trend is outsourcing a system's physical layer to a foreign party, for example, a \textit{cloud provider} --- this is beneficial in terms of cost and flexibility, but many security considerations assume that the physical layer itself can be trusted. This is not guaranteed when the physical layer is a \textit{virtual machine}, inflating the system's \textit{\acrshort{tcb}} with an external and transparent software layer, the underlying \textit{hypervisor}.

\paragraph{} \textit{\acrfull{tee}}, have long been explored by the security community as potential protection against this. They generate isolated processing contexts in which an operation can be securely executed irrespective of the rest of the system --- one example is software \textit{enclaves}. \textit{Enclaves} are general-purpose \textit{\acrshort{tee}} provided by the \acrshort{cpu}, protecting the logic found inside at the architectural level. Intel's \acrfull{sgx} is the most prolific example, affording a \textit{black-box} environment and runtime for arbitrary apps to execute under.

\paragraph{} An alternative approach is to use \textit{\acrfull{ifc}} to police system components. Enforced using a \textit{reference monitor}, \acrshort{ifc} models permissible data use, manipulating systems at a granular level.

\paragraph{} This work explores methods of hardening Linux with an \acrshort{sgx}-driven \textit{reference monitor} to track and protect host \acrshort{os} resources using \acrshort{ifc} methods. Further, it aims to reason what the future relationship between an \acrshort{os} and the enclaves it hosts should be, and whether complete isolation between them is the natural answer in several common situations.


\paragraph{Contributions}
\begin{itemize}
    \item \textsc{Citadel}, a prototype implementation of a modular \textit{reference monitor} protected using Intel \acrshort{sgx}, empowering \acrshort{ifc} techniques to operate with autonomy and protection from the host operating system. Enforcement is achieved using a \textit{\acrfull{lsm}} embedded in the Linux kernel, with an overall \textit{\acrshort{tcb}} of only a minimal footprint of the kernel alongside the enclave application.
    \item A userspace interposition library to near-transparently integrate unmodified applications to fully function under the new restrictions.
    \item A full port of the \textit{libtomcrypt} cryptography library for use inside an \acrshort{sgx} enclave.
    \item A rigorous investigation of performance implications, featuring the \textit{Nginx} production webserver. Worst-case performance shows a $24$\% decrease in request throughput, with other trials reporting performance parity with native Linux. Additionally we report a median overhead of $43\,\mu s$ (\acrshort{iqr} $26-72\,\mu s$, $n = 10^6$) per affected \textit{system call} without caching, matching or surpassing similar, non-enclave-based, systems.
\end{itemize}